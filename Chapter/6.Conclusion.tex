% !TEX root = ../MasterThesis_Onoe.tex
% 上記はただのコメントではなく親ファイルの場所を教えているので
% 消してしまうとファイルごとのタイプセットができなくなるので注意。
% 親ファイル名を変更したときはここも変更する。

\chapter{まとめと今後の展望} \label{sec:Conclusion}
本論文では、電子陽電子ヒッグスファクトリーのためのジェット測定技術の研究として、特にILCを念頭に置いた2つのテーマの研究を行った。1つ目のテーマでは、ILCの検出器案であるILDの電磁カロリメータのうち、SiW-ECALの技術プロトタイプをハドロンビーム試験によって性能評価した。2つ目のテーマでは、深層学習を用いたフレーバー識別アルゴリズムの開発を行った。本章ではそれぞれの研究のまとめと今後の展望について述べる。
\section*{ILD Si-W ECAL プロトタイプのハドロンビーム試験による性能評価}

\section*{深層学習を用いたフレーバー識別アルゴリズム}
本研究では、深層学習を用いたフレーバー識別アルゴリズムの開発を行った。ヒッグスファクトリーの目的であるヒッグス粒子は、クォークやグルーオンなどに崩壊し、多数のハドロンの束であるジェットとして検出器に到達する。そのためジェット再構成の一部であるフレーバー識別によってクォークのフレーバを識別することは、ヒッグスの物理を知るための解析において非常に重要な情報となる。フレーバー識別はジェットの構成粒子の種類や運動量、崩壊点に関する情報から、ジェットの元となるクォークのフレーバーを識別するプロセスであり、現在ジェットの再構成に用いられているLCFIPlusでは従来の機械学習手法であるBDTsが用いられている。これに対して深層学習を導入することで、識別性能の向上や崩壊点検出アルゴリズムとの統合などを目指した。\\
 はじめに、最も単純な構造のネットワークモデルであるディープニューラルネットワークによる実装を行った。
