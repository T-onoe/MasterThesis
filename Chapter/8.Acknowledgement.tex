% !TEX root = ../MasterThesis_Onoe.tex
% 上記はただのコメントではなく親ファイルの場所を教えているので
% 消してしまうとファイルごとのタイプセットができなくなるので注意。
% 親ファイル名を変更したときはここも変更する。

\clearpage

\chapter*{謝辞} \label{sec:Acknowledgement}
本研究を進めるにあたり、お世話になりましたすべての方々に感謝申し上げます。

本研究テーマは、指導教員である末原大幹助教に授けて頂きました。末原助教は、日々の研究において私の理解に寄り添い、多くの的確な助言を与えて下さいました。また、国内外での高エネルギー加速器実験への参加や学会・国際会議での発表など、新しいことに挑戦する機会もたくさん与えて頂きました。お陰様で貴重な経験を積み重ね、多くを学ぶことが出来ました。心より感謝致します。川越清以教授には、ゼミナールを通して素粒子物理学の基礎からILCに至るまで、素粒子実験の基礎を教えていただきました。また、研究の進捗報告や学会の発表練習においても多くの助言を頂き、本論文の執筆においても細かにご指導していただきました。厚くお礼申し上げます。
東城順治准教授には学部4年次の素粒子物理学の講義において、素粒子物理の基礎を教えて頂きました。吉岡瑞樹准教授には、学部3年次のプログラミングの講義を通してプログラミングによる数値計算の基礎を教えて頂きました。森津学助教には3年生実験のTA業務や、高校生の体験授業において大変お世話になり、親身になってご相談に乗って頂きました。音野瑛俊助教には、CERNにて最先端の実験施設や研究について教えて頂き、その他の面でもサポート頂きました。山中隆志助教には TA 業務で助言をいただいたほか、 g-2実験のASIC検査業務では普段の研究では出来ない経験をさせて頂きました。小林大元特任助教、細川律也学術研究員、小川真治特別研究員、水野貴裕学術研究員にはゼミナールや論文紹介の際に的確なご指摘やご指導を頂いたほか、研究室生活においても色々な話をする機会があり楽しく研究生活を過ごせました。テクニカルスタッフの重松さおり氏には、出張や研究室内の物品管理など研究に関わる事務作業において大変お世話になり、集中して研究に取り組むことが出来ました。また、理学部等事務部や物理学事務室の皆様には、TAや出張の際に事務手続きにおいてサポートいただきました。

ILCグループの先輩である後藤輝一氏には、深層学習の研究にあたって一部修士論文の内容を引き継いだこともあり、様々な面で参考にさせて頂きました。また同じくILCグループの先輩である久原真美氏には、研究における知識を丁寧に教えて頂いただけでなく、発表会や就職活動に至るまで様々なお話を聞かせて頂き、憧れの先輩でした。ILCグループの同輩にあたる津村周作君とは、互いに研究の進捗を切磋琢磨し、研究で行き詰まった時には納得がいくまで議論を交わすことのできる、貴重な友人でありました。ILCグループの後輩である永江航志君には、先輩として有益なアドバイスができる機会が少なかったことが悔やまれますが、研究に向き合う姿勢からは大変刺激を受けました。

研究室の先輩である高田秀佐氏、古賀淳氏、山口尚輝氏、宮崎祐太氏、野口恭平氏、竹内佑甫氏、 姚舜禹氏、松崎俊氏、岩下侑太郎氏には、論文紹介や月1ミーティングの研究進捗の報告など様々な場面でアドバイスを頂き、本論文を執筆する上でも多くの助言を頂きました。
同期に当たる谷田征輝君、宮本佳門君、樋口義清君、川上真言君とは、ゼミを通して議論を交わしたほか、研究以外にも私のつまらない話に付き合って頂き、楽しく大学院生活を送ることが出来ました。
後輩にあたる西原君、星野君、塩谷君、梅林君、山田君、Afiq Azraei君、周君、また特研生の土谷君、花田君、吉川君、今村君、水取君には、研究に取り組む姿に刺激を受けました。

また研究にあたって外部の研究機関の方々にも大変お世話になりました。大阪公立大学の岩崎昌子氏、大阪大学の中島悠太氏、長原一氏、九州工業大学の武村紀子氏には、分野外であった深層学習の研究に際して多くの助言を頂きました。高エネルギー加速器研究機構のDaniel Jeans氏、Junping Tian氏、東京学芸大学の日高啓晶氏には、物理解析に際して、また信州大学の竹下徹氏、東京大学の大谷航氏にはカロリメータの研究において多くの助言を頂きました。IJCLabのRoman P\''oschl氏、LLRのVincent Boundry氏、IFICのAdrian Irles氏、東北大学の奥川悠元氏には、CERNでのビームテスト実験において議論を交わし、カロリメータや解析について助言を頂きました。

その他多くの方にも本当にお世話になりました。皆様の助けあったからこそ研究を進めることが出来きました。

最後に、今まで24年間応援し支えてくれた両親に感謝の意を示し、謝辞の言葉とさせて頂きます。