% !TEX root = ../MasterThesis_Onoe.tex
% 上記はただのコメントではなく親ファイルの場所を教えているので
% 消してしまうとファイルごとのタイプセットができなくなるので注意。
% 親ファイル名を変更したときはここも変更する。

\begin{center}
%概要集に出すときにはタイトルや氏名が必要なので\iffalseを \ifture にする。
%本文で使用する場合は不要なので \iffalse にする。
\iffalse
\thispagestyle{empty}
{\Large 修士論文テンプレート}\\
九州大学大学院 理学府 物理学専攻 \\ 粒子物理学分野 素粒子実験研究室 \\
素粒子\ 実験 \\[1ex] 指導教員\ 氏\ 名\\   \\
\fi
{\huge 概要}\\
\end{center}
 2012年にLarge Hadron Collider(LHC)で発見されたヒッグス粒子は、素粒子の質量や宇宙の物質の起源の解明につながる粒子であり、その性質は謎に包まれている。そのためヒッグスファクトリーによってヒッグス粒子を大量に生成し、低バックグラウンドな環境で精密測定することで、ダークマターの候補になる新粒子探索などの標準理論を超える物理を切り開く足がかりとして期待されている。本研究で念頭に置いている国際リニアコライダー (International Linear Collider; ILC) 計画は次世代電子陽電子衝突型加速器であり、ヒッグスファクトリーとしての運転を期待されている。ヒッグスファクトリーの物理において重要な事象はジェットを含んでおり、ジェットを高い精度で測定・再構成することは、物理解析の性能に直結する。本研究ではジェット測定技術の研究として、検出器候補であるシリコンタングステン電磁カロリメータ (Sillicon Tangsten Electromagnetic Calorimeter; SiW-ECAL) の性能評価と、ジェット再構成におけるフレーバー識別の深層学習を用いたアルゴリズム開発を行なった。\\
 ILCで重要となるイベントはジェットを生成するが、ILCの目標感度を達成するためにはおおよそ ${\sigma}_E /E = 30\%\sqrt{E(GeV)}$の高いジェットエネルギー分解能が必要である。ILCではParticle Flow Algorithm (PFA) と呼ばれる粒子識別のアルゴリズムによって高い分解能の達成を目指しており、これには非常に高精細な電磁カロリメータが求められる。SiW-ECALはタングステンの吸収層とシリコンパッドセンサーの検出層が交互に積み重なったサンドウィッチ構造をしており、シリコンパッドセンサーは1つの読み出しセルの大きさが5mm$\times$5mmと非常に細分化されている。また、シリコンパッドはASICを搭載した読み出し基板に接着されており、PFAに最適化したコンパクトな構造となっている。現在日本やフランスを中心とした国際協力によって技術プロトタイプの開発が行われており、本論文ではその性能評価について記述する。特に、欧州原子核研究機構(CERN)のThe Super Proton Synchrotron(SPS)で行なった、高エネルギーハドロンビームを用いたテストビーム実験の結果から、読み出しシステムの性能やシリコンセンサーの振る舞いについて報告する。\\
 さらに、ILCのジェット再構成におけるフレーバー識別アルゴリズムの開発を、深層学習技術を用いて行った。フレーバー識別は、実験データからジェットを再構成する過程において、ジェットの軌跡や崩壊点に関する情報から、親粒子のフレーバーを識別するプロセスである。従来技術であるLCFIPlusでは、 フレーバー識別に機械学習技術であるBoosted Decision Trees(BDTs)が採用されている。近年の粒子物理学の分野では、物理解析やシミュレーションに深層学習を導入することで精度や性能を改善する取り組みが行われており、本研究においても識別性能の向上を目的に深層学習を実装した。本研究では特にグラフ形式で記述されたデータを学習に扱うグラフニューラルネットワーク (GNN) を使用した。GNNでは一般的なニューラルネットワークと比較して、物理現象の幾何学構造を活かし情報損失を減らすことができる。(本アルゴリズムの結果をLCFIPlusと比較したところ、現状ではより良い識別精度を得ることは出来なかったが、従来は別にプロセスを行なっていた崩壊点検出のアルゴリズムを統合することができた。)