% !TEX root = ../MasterThesis_Onoe.tex
% 上記はただのコメントではなく親ファイルの場所を教えているので
% 消してしまうとファイルごとのタイプセットができなくなるので注意。
% 親ファイル名を変更したときはここも変更する。

\ifabstract
 \maketitle
\fi
\begin{center}
{\LARGE 概要}\\
\end{center}
 2012年に発見されたヒッグス粒子は、宇宙の物質の起源の解明につながる粒子であり、その性質は謎に包まれている。そのため、ヒッグスファクトリーによってヒッグス粒子を大量に生成し精密測定することで、暗黒物質などの標準理論を超える物理を切り開くことは、現代の物理学における重要事項となっている。本研究で念頭に置いている国際リニアコライダー (ILC) 計画は次世代の電子陽電子衝突型加速器であり、ヒッグスファクトリーとしての運転を期待されている。ヒッグスファクトリーにおけるヒッグス粒子生成事象は、多数のハドロンの束であるジェットを終状態に複数含むことが多いため、ジェットを高い精度で検出・再構成することは物理解析の性能に直結する。そのため、本論文ではILCのジェット測定技術に関する2つの研究を行なった。

ILCで重要となる事象にはジェットを含むものが多くあるため、ILCの物理目標を達成するためには高いジェットエネルギー分解能が必要である。そのため、ILCではParticle Flow Algorithmと呼ばれる粒子識別のアルゴリズムによって高い分解能の達成を目指しており、これには非常に高精細な電磁カロリメータが求められる。シリコンタングステン電磁カロリメータは、1つの読み出しセルの大きさが$\SI{5}{mm} \times \SI{5}{mm}$と非常に細分化されており、国際協力によって技術プロトタイプの開発が行われている。本研究では、15層の検出層を持つプロトタイプを構築し、欧州原子核研究機構 (CERN) の Super Proton Synchrotron (SPS) 加速器にて最大$\SI{150}{GeV}$の高エネルギービームを用いたテストビーム実験を行った。そして実験結果から、読み出しシステムの性能やシリコンセンサーの振る舞いについて調査し、将来の技術プロトタイプに向けた改善策を講じた。特に読み出しにおいてトリガー情報が誤って収集されてしまうRe-triggering現象について調査を行った。

さらに、ILCのジェット再構成におけるフレーバー識別アルゴリズムの開発を、深層学習技術を用いて行った。フレーバー識別では、ジェットの構成粒子の種類や運動量、崩壊点に関する情報から、ジェットの元となるクォークのフレーバーを識別する。現在は従来の機械学習手法であるBoosted Decision Treesが用いられているが、本研究では更なる識別性能の向上を目的に深層学習を用いたアルゴリズムを開発した。本アルゴリズムの最大の特徴は、フレーバー識別のためにグラフ構造のデータを構築し、グラフデータの学習が可能なニューラルネットワークであるグラフニューラルネットワークで学習を行った点である。グラフデータはデータの関係をグラフ形式で表現することで、データの相互関係を考慮することができる。これによって実際の物理現象と比較した際の情報損失を減らし、より高い精度で識別を行うことが出来ると考えた。本アルゴリズムの結果は従来技術と比較して$b$フレーバーの識別のみの改善となったが、グラフデータの構築によって従来は別々にプロセスを行なっていた崩壊点検出のアルゴリズムを統合することができた。%--------2nd draft--------
% 2012年に発見されたヒッグス粒子は、宇宙の物質の起源の解明につながる粒子でありその性質は謎に包まれている。そのため、ヒッグスファクトリーによってヒッグス粒子を生成し精密測定することで、ダークマターなどの標準理論を超える物理を切り開くことが期待されている。本研究で念頭に置いている国際リニアコライダー (ILC) 計画は次世代電子陽電子衝突型加速器であり、ヒッグスファクトリーとしての運転を期待されている。ヒッグスファクトリーの物理において重要な事象はジェットを含んでおり、ジェットを高い精度で測定・再構成することは、物理解析の性能に直結する。本論文ではジェット測定技術の研究として、検出器候補であるシリコンタングステン電磁カロリメータ (SiW-ECAL) の性能評価と、ジェット再構成におけるフレーバー識別の深層学習を用いたアルゴリズムの開発を行なった。\\
% ILCで重要となるイベントはジェットを生成するが、ILCの物理目標を達成するためには高いジェットエネルギー分解能が必要である。そのため、ILCではParticle Flow Algorithmと呼ばれる粒子識別のアルゴリズムによって高い分解能の達成を目指しており、これには非常に高精細な電磁カロリメータが求められる。SiW-ECALのシリコンパッドセンサーは、1つの読み出しセルの大きさが$5\mathrm{mm} \times 5\mathrm{mm}$と非常に細分化されており、現在国際協力によって技術プロトタイプの開発が行われている。本研究では、15層の検出層を持つプロトタイプモジュールを構築し、欧州原子核研究機構 (CERN) のSPS加速器にて150$\mathrm{GeV}$の高エネルギービームを用いたテストビーム実験を行った。そして実験結果から、読み出しシステムの性能やシリコンセンサーの振る舞いについて調査し、将来の技術プロトタイプに向けた改善策を講じた。特に読み出しにおいてトリガー情報が誤って収集されてしまうRe-triggering現象や、センサーパッド一面を実際のヒットとは異なるヒットが満たしてしまうスクエアイベント現象について調査を行った。\\
% さらに、ILCのジェット再構成におけるフレーバー識別アルゴリズムの開発を、深層学習技術を用いて行った。フレーバー識別は実験データから粒子を再構成する過程の一部であり、ジェットの構成粒子の種類や運動量、崩壊点に関する情報から、ジェットの元となるクォークのフレーバーを識別する。現在は従来の機械学習手法である Boosted Decision Trees (BDTs) が用いられているが、本研究では更なる識別性能の向上を目的に深層学習を用いたアルゴリズムを開発した。本アルゴリズムの最大の特徴は、フレーバー識別のためにグラフ構造のデータを構築し、グラフデータの学習が可能なニューラルネットワークであるグラフニューラルネットワーク (GNN) で学習を行った点である。従来のBDTsやDNNでは、ジェットの運動量や崩壊点に関する情報のみで学習を行っている。一方で、グラフデータはデータの関係をグラフ形式で表現することで、データの相互関係を考慮することができる。これによって実際の物理現象と比較した際の情報損失を減らし、より高い精度で識別を行うことが出来ると考えた。本アルゴリズムの結果をLCFIPlusと比較したところ、現状ではbフレーバーの識別においてのみの改善となったが、グラフデータの構築によって従来は別々にプロセスを行なっていた崩壊点検出のアルゴリズムを統合することができた。
%------1st draft ----
%本アルゴリズムでは最も単純な構造である全結合層のみのニューラルネットワークに加えて、各ジェットに対して飛跡をノードとするグラフ構造のデータを構築し、隣接関係から重みパラメータを更新する畳み込みニューラルネットワーク (Graph Convolutional Network; GCN) によって学習を行った。GCNでは対象の特徴量に加えて、データ構造のトポロジーに物理現象の幾何学構造を反映させることが出来るため、情報損失を減らすことができると考えた。本アルゴリズムの結果をLCFIPlusと比較したところ、現状では一部の識別領域においてのみしか優れた結果を得られなかったが、従来は別々にプロセスを行なっていた崩壊点検出のアルゴリズムの一部を統合することができた。