% !TEX root = ../MasterThesis_Onoe.tex
% 上記はただのコメントではなく親ファイルの場所を教えているので
% 消してしまうとファイルごとのタイプセットができなくなるので注意。
% 親ファイル名を変更したときはここも変更する。

\begin{center}
%概要集に出すときにはタイトルや氏名が必要なので\iffalseを \ifture にする。
%本文で使用する場合は不要なので \iffalse にする。
\iffalse
\thispagestyle{empty}
{\Large 修士論文テンプレート}\\
九州大学大学院 理学府 物理学専攻 \\ 粒子物理学分野 素粒子実験研究室 \\
素粒子\ 実験 \\[1ex] 指導教員\ 氏\ 名\\   \\
\fi
{\huge 概要}\\
\end{center}
 2012年に発見されたヒッグス粒子は、宇宙の物質の起源の解明につながる粒子であり、その性質は謎に包まれている。そのためヒッグスファクトリーによってヒッグス粒子を大量に生成し精密測定することで、ダークマターなどの標準理論を超える物理を切り開くことが期待されている。本研究で念頭に置いている国際リニアコライダー (ILC) 計画は次世代電子陽電子衝突型加速器であり、ヒッグスファクトリーとしての運転を期待されている。ヒッグスファクトリーの物理において重要な事象はジェットを含んでおり、ジェットを高い精度で測定・再構成することは、物理解析の性能に直結する。本研究ではジェット測定技術の研究として、検出器候補であるシリコンタングステン電磁カロリメータ (SiW-ECAL) の性能評価と、ジェット再構成におけるフレーバー識別の深層学習を用いたアルゴリズムの開発を行なった。\\
 ILCで重要となるイベントはジェットを生成するが、ILCの物理目標を達成するためには高いジェットエネルギー分解能が必要である。ILCではParticle Flow Algorithm (PFA) と呼ばれる粒子識別のアルゴリズムによって高い分解能の達成を目指しており、これには非常に高精細な電磁カロリメータが求められる。SiW-ECALにおけるシリコンパッドセンサーは、1つの読み出しセルの大きさが5mm$\times$5mmと非常に細分化されており、現在日本やフランスを中心とした国際協力によって技術プロトタイプの開発が行われている。本研究では、15層の検出層を持つプロトタイプモジュールを構築し、CERNのSPS加速器にて150$\mathrm{GeV}$の高エネルギーハドロンビームを用いたテストビーム実験を行った。実験結果から、読み出しシステムの性能やシリコンセンサーの振る舞いについて調査し、将来の技術プロトタイプに向けた改善策を講じた。特に読み出しにおいてトリガー情報が誤って収集されてしまうRe-triggering現象や、センサーパッド一面を実際のヒットとは異なるヒットが満たしてしまうスクエアイベント現象について調査を行った。\\
 さらに、ILCのジェット再構成におけるフレーバー識別アルゴリズムの開発を、深層学習技術を用いて行った。フレーバー識別は、実験データからジェットを再構成する過程において、ジェットの軌跡や崩壊点に関する情報から親粒子のフレーバーを識別するプロセスである。フレーバー識別はILCの物理プログラムにおいて重要なプロセスであり、従来技術であるLCFIPlusではフレーバー識別に人がパラメータを調整する機械学習技術のBDTが採用されているが、本研究では更なる識別性能の向上を目的に深層学習を用いたアルゴリズムを開発した。本研究では特にグラフ形式で記述されたデータを学習に扱うグラフ畳み込みニューラルネットワーク (GCN) を使用した。GCNでは単純なニューラルネットワークと比較して、物理現象の幾何学構造を活かし情報損失を減らすことができる。本アルゴリズムの結果をLCFIPlusと比較したところ、現状ではより良い識別精度を得ることは出来なかったが、従来は別でプロセスを行なっていた崩壊点検出のアルゴリズムの一部を統合することができた。