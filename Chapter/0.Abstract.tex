% !TEX root = ../MasterThesis_Onoe.tex
% 上記はただのコメントではなく親ファイルの場所を教えているので
% 消してしまうとファイルごとのタイプセットができなくなるので注意。
% 親ファイル名を変更したときはここも変更する。

\begin{center}
%概要集に出すときにはタイトルや氏名が必要なので\iffalseを \ifture にする。
%本文で使用する場合は不要なので \iffalse にする。
\iffalse
\thispagestyle{empty}
{\Large 修士論文テンプレート}\\
九州大学大学院 理学府 物理学専攻 \\ 粒子物理学分野 素粒子実験研究室 \\
素粒子\ 実験 \\[1ex] 指導教員\ 氏\ 名\\   \\
\fi
{\huge 概要}\\
\end{center}
\ 本研究では、電子陽電子ヒッグスファクトリーにおいて、電子陽電子の衝突により発生するジェット測定技術の研究として2つの技術研究を行った。\\
\ 2012年にLarge Hadron Collider(LHC)での実験によって発見されたヒッグス粒子は、素粒子の質量や宇宙の物質の起源の解明につながる粒子であり、その性質は謎に包まれている。そのためヒッグスファクトリーによってヒッグス粒子を大量に生成し、低バックグラウンドな環境で精密測定することで、ダークマターの候補になる新粒子探索などの標準理論を超える物理を切り開く足がかりとして期待されている。本研究において扱う国際リニアコライダー(International Linear Collider:ILC)計画は次世代電子陽電子衝突型加速器であり、ヒッグスファクトリーとしての運転を期待されている。本論文では、その検出器案であるシリコンタングステン電磁カロリメータ Sillicon Tangsten Electromagnetic Calorimeter (SiW-ECAL) の性能評価と、ILCのジェット再構成におけるフレーバー識別のアルゴリズム開発に関する研究を行なった。\\
\ ILCで重要となるイベントにはジェットが含まれ、ILCの目標感度を達成するためにはおおよそ ${\sigma}_E /E = 30\%/\sqrt(E)(GeV)$のエネルギー分解能を達成する必要がある。そこでILCはParticle Flow Algorithm (PFA) と呼ばれる粒子識別のアルゴリズムを使用し、これには非常に高精細な電磁カロリメータが必要となる。SiW-ECALはタングステンの吸収層とシリコンパッドセンサーの検出層が交互に積み重なったサンドウィッチ構造をしており、シリコンパッドセンサーは1つの読み出しセルの大きさが5mm$\times$5mmと非常に細分化されている。また、シリコンパッドはASICを搭載した読み出し基板に接着されており、PFAに最適化したコンパクトな構造となっている。現在日本やフランスを中心とした国際協力によって技術プロトタイプの開発が行われており、本論文ではその性能評価について記述する。特に、欧州原子核研究機構(CERN)のThe Super Proton Synchrotron(SPS)で行なった高エネルギーハドロンビームを用いたビームテストの結果から、読み出しシステムの性能やシリコンセンサーの振る舞いについて報告する。\\
\ さらにジェット再構成においてフレーバー識別はジェット再構成において、ジェットの軌跡や崩壊点に関する情報から、親粒子のフレーバーを識別する過程である。現在ILCでジェット再構成に使用されているLCFIPlusというソフトウェアでは、 フレーバー識別に機械学習技術の一つであるBoosted Decision Trees(BDTs)が採用されている。近年、粒子物理学の分野では物理解析やシミュレ ーションに深層学習を使用し、精度や性能を改善する取り組みが行われており、本研究においても深層学習の適用によって識別性能の向上を。本研究では特にグラフ形式で記述されたデータを学習に扱うグラフニューラルネットワーク(GNN)について、\\
(未完成)

\cite{sample}