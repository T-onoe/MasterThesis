% !TEX root = ../MasterThesis_Onoe.tex
% 上記はただのコメントではなく親ファイルの場所を教えているので
% 消してしまうとファイルごとのタイプセットができなくなるので注意。
% 親ファイル名を変更したときはここも変更する。

\begin{center}
%概要集に出すときにはタイトルや氏名が必要なので\iffalseを \ifture にする。
%本文で使用する場合は不要なので \iffalse にする。
\iffalse
\thispagestyle{empty}
{\Large 修士論文テンプレート}\\
九州大学大学院 理学府 物理学専攻 \\ 粒子物理学分野 素粒子実験研究室 \\
素粒子\ 実験 \\[1ex] 指導教員\ 氏\ 名\\   \\
\fi
{\huge 概要}\\
\end{center}

本研究では、電子陽電子ヒッグスファクトリーにおいて衝突により発生するジェット測定技術の研究として、2つの研究を行った。国際リニアコライダー(International Linear Collider:ILC)計画は次世代電子陽電子衝突型加速器であり、その検出器案であるシリコンタングステン電磁カロリメータ Sillicon Tangsten Electromagnetic Calorimeter (SiW-ECAL) の性能評価に関する研究を行った。
ILC は現在建設が検討されている全長約 20 kmの線形加速器で、電子と陽電子の衝突によって生じるジェットの解析により Higgs 粒子の精密測定やダークマターの候補となる新粒子の探索などが可能とされ、標準理論を超える物理を切り拓く足がかりとして期待されている。ILC で重要となるイベントにはジェットが含まれ、ILC の目標感度を達成するためにはおおよそ

\cite{sample}