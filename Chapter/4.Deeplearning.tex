% !TEX root = ../MasterThesis_Onoe.tex
% 上記はただのコメントではなく親ファイルの場所を教えているので
% 消してしまうとファイルごとのタイプセットができなくなるので注意。
% 親ファイル名を変更したときはここも変更する。

\chapter{深層学習} \label{sec:Deeplearning}
本章では、本研究で提案する手法の基礎となる深層学習の理論を述べる。
\section{ニューラルネットワーク}
\subsection{パーセプトロン(単層ニューラルネットワーク)}
ニューラルネットワークの基礎となるパーセプトロンは、ローゼンブラットにより1957年に考案された。パーセプトロンの基本構造は、信号を入力として受け取り論理回路を通して出力信号を出すものである。図\ref{perceptron}に最も基本的なパーセプトロンの例を示す。$x_1, x_2$は入力信号、$y$が出力信号であり、$w_1,w_2$がそれぞれの入力信号にかかる重みを表す。また、図中における$\bigcirc$はノードと呼ぶ。入力信号はノードに送られる前に重みが掛けられ、出力ノードにてそれらの総和をとる。出力ノードでの演算(活性化関数)をステップ関数(階段関数)とすると、その総和が閾値$\theta$を超えている場合のみ出力信号は1を出力することになる。数式で示すと以下のようになる。\\
\begin{align}
 y =
 \begin{cases}
 0 & (w_1x_1 + w_2x_2 ) \leq \theta\\
 1 & (w_1x_1 + w_2x_2 ) > \theta \\
 \end{cases}
\end{align}
 パーセプトロンにおいて重要となるのは入力信号に対する固有の重みであり、重みは各信号の重要性を操作する要素として働く。すなわち重みが大きいほど、対応する信号の全体における重要性が高くなる。この重みを更新する操作を学習と呼び、ニューラルネットワークでは学習を繰り返すことで重みパラメータを理想とする値に近づけていく。\\
 また、入力信号が2つ以上の場合についても考えることができ、以下のような式で表される。入力信号$x = \{ x_1, x_2, \ldots x_n \}$、重みパラメータ$w = \{ w_1, w_2, \ldots w_n \}$、活性化関数(ここではステップ関数)を$h(x)$とすると、出力ベクトルyは
\begin{align}
y = h(w^T x) =
 \begin{cases}
 0 & (w_1x_1 + w_2x_2 + \ldots + w_nx_n) \leq \theta\\
 1 & (w_1x_1 + w_2x_2 + \ldots + w_nx_n) > \theta \\
 \end{cases}
\end{align}
\subsection{多層パーセプトロン(多層ニューラルネットワーク)}
パーセプトロンの演算では線形領域のみしか表現できず、非線形領域においても扱えるよう入力層と出力層の間に中間層(隠れ層)を加えるニューラルネットワークに改良された。このような中間層を複数重ねたパーセプトロンを多層パーセプトロン(Multi Layer Perceptron, MLP)と呼ぶ。多層パーセプトロンの簡単な例を図\ref{mlp}に示す。最も左のノード列を入力層、真ん中のノード列を中間層、一番右のノード列を出力層とすると、以下のような数式で表される。入力信号$x^0 = \{ x_1^0, x_2^0, \ldots x_n^0 \}$、中間層の各ノードに入ってくる信号$x^1 = \{ x_1^1, x_2^1, \ldots x_n^1 \}$、入力層と中間層の信号にかかる重みパラメータがそれぞれ$w^0 = \{ w_1^0, w_2^0, \ldots w_n^0 \}, w^1 = \{ w_1^1, w_2^1, \ldots w_n^1 \}$、ステップ関数の活性化関数を$h(x)$とすると、出力ベクトル$y_n$は
\begin{align}
y_n = h(w^T )
 \begin{cases}
 \end{cases}
\end{align}
となる。これはそれぞれに対して行列を用いると、以下のように簡略に表現できる。\\
\begin{align}
not yet
\end{align}
 以下ではニューラルネットワークの学習における、学習の仕組みや重要な技術について取り上げる。
\subsubsection{活性化関数}
活性化関数はニューラルネットワークにおける入力の重み線形和から、出力を決定するための関数である。活性化関数には主に非線形関数が用いられ、以下に主なものについて示す。
\begin{itemize}
	\item ステップ(階段)関数
		\begin{align}
			h(a) =
			\begin{cases}
			0 & (a \leq \theta)\\
			1 & (a > \theta)
			\end{cases}
		\end{align}
	\item sigmoid関数
		\begin{equation}
			h(a) = \frac{1}{1+exp(-a)}
		\end{equation}
	\item $tanh$関数
		\begin{equation}
			h(a) = tanh(a)
	\end{equation}
	\item ReLU関数(ランプ関数)
		\begin{align}
			h(a) =
			\begin{cases}
			0 & (a \leq \theta)\\
			a & (a > \theta)
			\end{cases}
		\end{align}
\end{itemize}

\subsubsection{出力層の設計}
ニューラルネットワークで扱える問題は、主に回帰問題と分類問題に分けられる。それぞれの問題によって出力層の設計が異なり、回帰問題においては恒等関数が、分類問題においてはソフトマックス関数が用いられる。恒等関数では入力された値をそのまま出力する。ソフトマックス関数は以下の式\ref{softmax}のように、0から1までの値を出力する関数であり、それぞれのカテゴリに分類される確率を表す。ここで$y_k$はニューラルネットワークの出力、$x_k$は出力層へ入ってくる信号を表す。また、出力層のノード数は問題に合わせて適宜調整する必要があり、分類問題であればカテゴリ数だけノードを設計する必要がある。
\begin{align}
y_k = \frac{exp(x_k)}{\sum_{i=1}^n exp(x_i)}
\end{align}

\subsubsection{損失関数}
先述の通り、ニューラルネットワークでは学習によって重みを更新するが、その際に学習結果を正しい答えと照らし合わせて評価し、重みを更新する。この評価関数を損失関数 (Loss function) と呼ぶ。損失関数には主に以下の2つが用いられる。
\begin{itemize}
	\item 二乗和誤差関数  (Mean Squared Error) \\
		二乗和誤差関数は、以下の式\ref{mse}で定義される関数である。ここで、$yk$はニューラルネットワークの出力、$t_k$は正解ラベルを表し、kはデータの次元数を表す。二乗和誤差関数の微分値はyの一次関数となっていることから、出力・正解ラベルが共に連続値であり恒等関数を出力層に持つ回帰問題で採用される。
		\begin{align}
			\label{mse}
			E = \frac{1}{2}\sum_k {(y_k-t_k)}^2
		\end{align}
	\item 交差エントロピー誤差 (Cross Entropy Error) 
		交差エントロピー誤差は、以下の式\ref{cee}で定義される関数である。ここで、$y_k$はニューラルネットワークの出力で、$t_k$はone-hot表現の正解ラベルを表す。交差エントロピー誤差は主にソフトマックス関数を出力層に用いる分類問題において採用される。
		\begin{align}
			\label{cee}
			E = - \sum_k t_k log(y_k)
		\end{align}
\end{itemize}

\subsubsection{誤差逆伝播法}
これまではニューラルネットワークの順方向の伝播(forward propagation)について見てきたが、出力層において学習結果と正解ラベルを比較し、逆方向に信号を伝播させ重みを更新するアルゴリズムを誤差逆伝播法 (back propagation) という。誤差逆伝播法では、次のような処理を行う。
\begin{enumerate}
	\item ニューラルネットワークにおいて順方向に学習を行い、出力層で損失関数によって正解ラベルとの誤差を求める。
	\item 誤差から各出力層ノードについて期待される出力と重要度、誤差を計算する。これを局所誤差という。
	\item 特に重要度の高い前層の入力が、局所誤差に影響を及ぼしているとして重みを調整する。
	\item さらに前層へと処理を繰り返す。
\end{enumerate}
\subsubsection{ミニバッチ処理}
ニューラルネットワークを学習させるにあたって、データを1つ1つ学習させるわけではない。実際に はミニバッチと呼ばれる、トレインデータをいくつかまとめて束として学習させる。この束をミニバッチ という。また、このミニバッチのサイズ、つまりいくつのデータをまとめて束にするかという値のことを バッチサイズという。
数値計算を扱うライブラリの多くは、大きな配列の計算を効率よく処理できるよう最適化がなされていおり、ミニバッチによる学習を行うことで、処理時間を短縮することができる。
\subsubsection{最適化アルゴリズム}
ニューラルネットワークの学習では、損失関数の値が最小となるような最適なパラメータを探索する。しかし損失関数のパラメータ空間は非常に複雑であることから、最適化は難しい。以下では、勾配降下法をはじめとする最適化手法について述べる。また、一度の学習で更新するパラメータの度合いを学習率 (learning rate) と呼んでおり、ネットワークの重みなどのパラメータとは異なり、学習率のような人の手で設定する必要のあるパラメータをハイパーパラメータと呼ぶ。
\paragraph{勾配降下法}
現在のネットワークのパラメータの微分 (勾配) を計算し、その微分の値を手がかりにパラメータの値を徐々に更新する方法を、勾配降下法 (gradient descent method) という。勾配降下法は以下の式\ref{gd}のように表される。ここで、$\eta$は学習率を表す。\\
\begin{align}
 \label{gd}
 x_0 = x_0 - \eta \frac{\partial f}{\partial x_0}\\
 x_1 = x_1 - \eta \frac{\partial f}{\partial x_1}\\
\end{align}
 また、後述のミニバッチ学習を用いた勾配降下法は特に、確率的勾配降下法 (stocastic gradient descent, SGD) と呼ばれており、現在のニューラルネットワークの最適化法は主にSGDに基づいて設計されている。一方で、SDGには関数の形状が等方的でない場合、勾配の方向が最終的な最小値と異なるため探索が非効率になるという欠点があり、単純に勾配方向へ進む以外の方法としてさまざまな最適化手法が考案されている。
\paragraph{モーメンタム}
モーメンタム (Momentum) は、それまでの学習における損失関数上で更新ステップの動きを考慮することでSGDの振動を抑えるアルゴリズムである。
\paragraph{AdaGrad}
AdaGradでは、モーメンタムと同様にSGDの振動を抑えるが、学習率を減衰させることによってこれを達成するアルゴリズムである。
\paragraph{Adam}
モーメンタム+AdaGrad
\subsection{ディープニューラルネットワーク}
多層ニューラルネットワークにおいて、特に層の数を多数持つモデルに対しては深い(ディープな)ことからディープニューラルネットワーク(Deep Neural Network, DNN, 深層学習)と呼ぶ。ディープニューラルネットワークの演算には過去にいくつかの技術的課題が存在していたが、計算機性能の向上に加え以下に挙げる計算技術の工夫などによって学習が可能となった。

\section{グラフニューラルネットワーク}

\subsection{Graph Convolution Network (GCN)}

\subsubsection{Spectral Graph Convolution}

\subsubsection{Spatial Graph Convolution}

\subsection{Graph Attention Network (GAT)}

\subsection{グラフニューラルネットワークの応用}