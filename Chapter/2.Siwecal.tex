% !TEX root = ../MasterThesis_Onoe.tex
% 上記はただのコメントではなく親ファイルの場所を教えているので
% 消してしまうとファイルごとのタイプセットができなくなるので注意。
% 親ファイル名を変更したときはここも変更する。

\chapter{シリコンタングステン電磁カロリメータ} \label{sec:1.Siwecal}
本章では、カロリメータの検出原理やシリコンタングステン検出器の検出原理と読み出し方法、またASICの設計性能やSi W- ECAL の読み出し方法、現在の技術プロトタイプについて説明する。
\section{入射粒子と物質の相互作用}
素粒子実験で捉えたい素粒子やハドロンは、粒子と物質との相互作用によって捉えることができる。本節では入射粒子の種類ごとに物質との相互作用について述べる。
\subsection{荷電粒子}
荷電粒子は物質を通過するとき、物質中の原子が持つ電子との非弾性散乱、あるいは原子核との弾性散乱を起こす。これによって荷電粒子はエネルギーを失う、あるいは進行方向を変える。
物質中の原子をイオン化し、エネルギーを失う。電離によるエネルギー損失は、Bethe-Blochの式に従う。
(記入途中)
\begin{equation}
-\frac{dE}{dx} = 2\pi N_a {r_e}^2m_ec^2\rho \frac{Z}{A} \frac{z^2}{{\beta}^2}[ln(\frac{2m_e{\gamma}^2v^2W_{max}}{I^2}) -2{\beta}^2 - \delta 2\frac{C}{Z} ]
\end{equation}

\subsection{光子}
中性粒子である光子は電離を起こさず、よって連続的に飛跡を検出することはできない。光子の相互作用としては図\ref{comptons}に示す光電効果、コンプトン散乱、電子陽電子対生成の 3 つが主に挙げられる。光電効果とは、光が物質に当たることで光の持っていたエネルギーが物質の電子に与えられ、励起された電子が飛び出す現象である。コンプトン散乱は入射光子と原子核に束縛されている電子との弾性散乱であり、電子陽電子対生成過程では原子のクーロン場において入射光子が消失し電子陽電子を生成する。光子のエネルギーによってこれらの反応確率は異なり、図\ref{photon}に光子のエネルギーに対する各反応の確率を示す。中でも約10MeV 以上の光子においては電子陽電子生成反応が主要なプロセスであり、ILCのような高エネルギーにおいては電子陽電子対生成が重要である。光子は測定器で直接検出出来ないが、このプロセスにより電子陽電子に変換することで、上述の電離により検出ができるようになる。物質中での光子の飛程は前述の $X_0$ を用いると、約 $9/7 X_0 $である。重い物質中に入射した光子は、電子陽電子に変換し、 それらが制動輻射で光子を発生させる。このプロセスが繰り返されることで電子・光子数が指数関数的に増加していき、これを電磁シャワーと呼ぶ。電磁シャワー発展するにつれ各粒子のエネルギーは低下していき、10 MeVを下回ると電子陽電子生成が不可能になり、シャワーは収束する。シャワー中に生成する電子・陽電子の数は元の光子のエネルギーに比例するため、これらをMinimum Ionization Particle (MIP) と見なし、シャワー中に設置したセンサーに残すエネルギーの和をとることで元の粒子のエネルギーを測定できる。
\subsection{ハドロン}
ハドロンは、原子核と衝突して非弾性散乱することで複数のハドロンを生成する。これが連鎖的に起きると電磁シャワーと同様のシャワーが起き、これをハドロンシャワー と呼ぶ。ハドロンの相互作用長は電子や光子と比べ非常に長く、ハドロンをカロリメータで測定するには多くの物質が必要となる。また非弾性散乱では破砕された物質の質量も反応に寄与するためシャワーの統計ゆらぎは更に大きくなる。従ってハドロンシャワーのエネルギー分解能は非常に悪くなってしまう。
\section{シリコンタングステン電磁カロリメータ SiW-ECAL}
\subsection{シリコン半導体検出器}
シリコン半導体検出器は、半導体を用いた放射線検出器のうちシリコンを用いた検出器である。基本構造は、板状のシリコン半導体の片面に P 型半導体、もう片面にN型半導体を形成しており、逆バイアス電圧をかけることで、内部に空乏層と呼ばれる伝導体にキャリアがない層を形成する。空乏層に荷電粒子が入射すると、飛跡に沿ってシリコン原子が電離し、電子正孔対が生成される。電子正孔対は空乏層内の電界によって運ばれるため電流が発生し、それを測定することで入射した荷電粒子のエネルギー損失を測定することができる。シリコンでは1つの電子ホール対を生成するのに約 3.6 eVのエネルギー損失が必要であり、センサー の有効厚を約 300nm とすると、MIP の通過によ り生成される電子対は約 24000, 電荷にして3.8fC程度となる。シリコン半導体検出器は、時間応答性がはやくエネルギー分解能が優れているため、ILCでは飛跡検出器とカロリメータにおいて使用されている。
\subsection{SiW-ECALの構造}
\section{読み出しシステム}
\subsection{SKIROC2A}
\subsubsection{Fast shaper}
\subsubsection{Slow shaper}
\subsection{FPGA}
\subsection{多層読み出しモジュール}

\section{技術プロトタイプ}