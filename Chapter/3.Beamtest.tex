% !TEX root = ../MasterThesis_Onoe.tex
% 上記はただのコメントではなく親ファイルの場所を教えているので
% 消してしまうとファイルごとのタイプセットができなくなるので注意。
% 親ファイル名を変更したときはここも変更する。

\chapter{ビームテストによる評価実験} \label{sec:Beamtest}
これまでに作成されたSiW-ECALの技術プロトタイプ(FEVおよびCOB)の性能評価実験を、2023年6月7日から2023年6月22日の期間にCERN SPS加速器のビームラインにて行った。本実験の主な目的は、電磁カロリメータとハドロンカロリメータの技術プロトタイプを同じビーム軸上に設置し、同時に運転を行いデータを取得すること。また、これまでの評価実験の中でも最高エネルギーのハドロンビームを用いて15層のSiW-ECALの評価を行うことの2点であった。また本実験におけるハドロンカロリメータは、同じくCALICEグループにおいてドイツやチェコが中心となって開発を進めているAHCALを用いた。以下では実験の詳細と、結果について述べる。
\section{CERN SPS}

\section{実験セットアップ}
\subsection{測定機器のセットアップ}
\subsection{EUDAQによる信号読み出し}
\section{実験結果}
\subsection{検出器応答}
\subsection{ペデスタル}
\subsection{スクエアイベント}
\section{まとめと考察}